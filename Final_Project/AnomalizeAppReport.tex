\documentclass[]{article}
\usepackage{lmodern}
\usepackage{amssymb,amsmath}
\usepackage{ifxetex,ifluatex}
\usepackage{fixltx2e} % provides \textsubscript
\ifnum 0\ifxetex 1\fi\ifluatex 1\fi=0 % if pdftex
  \usepackage[T1]{fontenc}
  \usepackage[utf8]{inputenc}
\else % if luatex or xelatex
  \ifxetex
    \usepackage{mathspec}
  \else
    \usepackage{fontspec}
  \fi
  \defaultfontfeatures{Ligatures=TeX,Scale=MatchLowercase}
\fi
% use upquote if available, for straight quotes in verbatim environments
\IfFileExists{upquote.sty}{\usepackage{upquote}}{}
% use microtype if available
\IfFileExists{microtype.sty}{%
\usepackage{microtype}
\UseMicrotypeSet[protrusion]{basicmath} % disable protrusion for tt fonts
}{}
\usepackage[margin=1in]{geometry}
\usepackage{hyperref}
\hypersetup{unicode=true,
            pdftitle={AnomalizeApp Instructions},
            pdfborder={0 0 0},
            breaklinks=true}
\urlstyle{same}  % don't use monospace font for urls
\usepackage{graphicx,grffile}
\makeatletter
\def\maxwidth{\ifdim\Gin@nat@width>\linewidth\linewidth\else\Gin@nat@width\fi}
\def\maxheight{\ifdim\Gin@nat@height>\textheight\textheight\else\Gin@nat@height\fi}
\makeatother
% Scale images if necessary, so that they will not overflow the page
% margins by default, and it is still possible to overwrite the defaults
% using explicit options in \includegraphics[width, height, ...]{}
\setkeys{Gin}{width=\maxwidth,height=\maxheight,keepaspectratio}
\IfFileExists{parskip.sty}{%
\usepackage{parskip}
}{% else
\setlength{\parindent}{0pt}
\setlength{\parskip}{6pt plus 2pt minus 1pt}
}
\setlength{\emergencystretch}{3em}  % prevent overfull lines
\providecommand{\tightlist}{%
  \setlength{\itemsep}{0pt}\setlength{\parskip}{0pt}}
\setcounter{secnumdepth}{0}
% Redefines (sub)paragraphs to behave more like sections
\ifx\paragraph\undefined\else
\let\oldparagraph\paragraph
\renewcommand{\paragraph}[1]{\oldparagraph{#1}\mbox{}}
\fi
\ifx\subparagraph\undefined\else
\let\oldsubparagraph\subparagraph
\renewcommand{\subparagraph}[1]{\oldsubparagraph{#1}\mbox{}}
\fi

%%% Use protect on footnotes to avoid problems with footnotes in titles
\let\rmarkdownfootnote\footnote%
\def\footnote{\protect\rmarkdownfootnote}

%%% Change title format to be more compact
\usepackage{titling}

% Create subtitle command for use in maketitle
\newcommand{\subtitle}[1]{
  \posttitle{
    \begin{center}\large#1\end{center}
    }
}

\setlength{\droptitle}{-2em}

  \title{AnomalizeApp Instructions}
    \pretitle{\vspace{\droptitle}\centering\huge}
  \posttitle{\par}
    \author{}
    \preauthor{}\postauthor{}
    \date{}
    \predate{}\postdate{}
  

\begin{document}
\maketitle

\subparagraph{Introduction}\label{introduction}

This document will explain how to use the AnomalizeApp located at
\url{https://rejohn.shinyapps.io/AnomalizeApp/}.

\subparagraph{About the Data}\label{about-the-data}

The app uses data from the \texttt{tibbletime} package
called\texttt{FANG}, an acronym created from the stock names for the
companies Facebook, Amazon, Netflix, and Google.

\subparagraph{Purpose}\label{purpose}

The primary purpose of the \texttt{anomalize} package is to detect
anomalies in time series data. In this application, I will demonstrate
how the \texttt{anomalize} package helps to detect anomalies in stock
data.

\subparagraph{Time Series Data Panel}\label{time-series-data-panel}

You can control what data you wish to detect anomalies in: * The first
drop down menu, ``Stocks'', you can pick which company's stocks you want
to detect anomalies in. * The second drop down menu, ``Detect Anomalies
in:'' allows you to control what type of stock data you want to detect
anomalies in.

\subparagraph{Decomposition Tab}\label{decomposition-tab}

In the decomposition tab, you can control the \texttt{method} parameter
for the \texttt{time\_decompose()} function to change how the function
decomposes the time series data in the ``Decomposition Method'' drop
down menu.

\subparagraph{Anomaly Detection}\label{anomaly-detection}

In the anomaly detection tab, you can control three parameters for the
\texttt{anomalize()} function: * \texttt{method}: Controls which method
to use to detect anomalies in the decomposed time series data with the
``Anomaly Detection Methods'' drop down menu. * \texttt{alpha}: Controls
the range of what is considered a ``normal'' value which the ``Alpha''
slider. The larger the value, the more easier it is for a data point to
be considered an anomaly. * \texttt{max\_anoms}: Controls the maximum
amount of the data that can be considered an anomaly with the ``Maximum
Percentage of Anomalies Allowed'' slider (uses decimal values).

\subparagraph{General Plot Aesthetics}\label{general-plot-aesthetics}

In the General Plot Aesthetics tab, you can control four parameters for
the \texttt{plot\_anomalies()} and
\texttt{plot\_anomaly\_decomposition()} functions to control the
appearance of the graphs displayed in the ``Anomaly Plot'' and
``Decomposition Plot'' tabs: \emph{\texttt{color\_no}: Controls the
color of ``normal'' points on the graph with the ``Dot Color'' color
picker. }\texttt{color\_yes}: Controls the color of anomalies on the
graph with the ``Anomaly Color'' color picker.
\emph{\texttt{alpha\_dots}: Controls the transparency of the points on
the graph with the ``Dot Alpha'' slider. }\texttt{size\_dots}: Controls
the size of the points on the graph with the ``Dot Size'' slider.

\subparagraph{Plot-Specific Aesthetics}\label{plot-specific-aesthetics}

In the Plot-Specific Aesthetics tab, you can control a parameter
specific to \texttt{plot\_anomalies()} and a parameter specifc to
\texttt{plot\_anomaly\_decomposition()}: \emph{\texttt{fill\_ribbon}: A
parameter in \texttt{plot\_anomalies()}. Controls the color of the band
of ``normal'' values in the Anomaly Plot with the ``Band Color'' color
picker. }\texttt{strip.position}: A parameter in
\texttt{plot\_anomaly\_decomposition()}. Controls where the label is in
the Decomposition Plot with the ``Label Position'' drop down menu.


\end{document}
